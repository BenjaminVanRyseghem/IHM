 \documentclass[a4paper,10pt]{article}
\input{/Users/WannaGetHigh/workspace/latex/macros.tex}

\title{IHM - TP: jTunes}
\author{Fran�ois \bsc{Lepan} Benjamin \bsc{Van Ryseghem}}

\begin{document}
\maketitle

\section*{Introduction}

Ce rapport d�crit la conception et l'utilisation d'un lecteur mp3. Nous verrons tout d'abord comment utiliser ce lecteur et ensuite nous verrons la conception de celui-ci.

\section{Le lecteur jTunes}


\begin{figure}[ht]
\begin{center}
	\includegraphics[width=15cm]{}
\end{center}
	\caption{}
	\label{lecteur_mp3}
\end{figure}

Ce lecteur (\emph{c.f.} ~Fig.~\ref{fig:}) est compos� d'un curseur permettant de savoir ou en est la music lors de la lecture et de changer la position de la lecture. � cot� de ce curseur on retrouve des labels indiquant la dur�e de la music, le temps �coul�  ainsi que le nom de la chanson jou�.

En dessous on retrouve les boutons "play/pause", "next" et "previous" qui permettent de changer les musiques jou�es et a cot� de ceux-ci se trouve le curseur permettant de g�rer le son.

� droite de ces �l�ments on retrouve la barre de recherche permettant comme son nom l'indique de rechercher des musiques dans la base de donn�e. En dessous de cette barre de recherche on retrouve deux boutons, "undo" et "redo" permettant d'annuler ou refaire la derni�re action.

En dessous de ces �l�ments on retrouve la liste de music, qui affiche les informations des titres pr�sent dans la base de donn�e.

\section{Impl�mentation}

Ce lecteur mp3 � �t� con�u sur la base d'un MVC. Sur l'UML (\emph{c.f.} ~Fig.~\ref{})

\begin{figure}[ht]
\begin{center}
	\includegraphics[width=15cm]{}
\end{center}
	\caption{}
	\label{uml}
\end{figure}


\section*{Conclusion}

\end{document}